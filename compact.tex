\section{Compact analysis objects}
To speed up the analysis process cycle and to optimise the storage, the CMS and ATLAS collaborations are transitioning towards more compact datasets for analysis with event sizes in the order of the kB/event. Based on the numbers for compact analysis objects (nanoAOD \cite{nano}) provided by CMS, a full analysis dataset will be close to 1~PB per year. This is largely lower than the 50~PB per year for older analysis objects (miniAOD)\footnote{ these sizes have been estimated taking as a reference LHC delivery of 80 billion events/year (data) and the production of 160 billion events/year (MC) together with expected sizes for the different data types of 7.4~MB(RAW), 2.0~MB(AOD), 200~kB(miniAOD) and 4~kB(nanoAOD)}. Compact objects open the window to evaluate new ways to address the user analysis challenge and propose different scenarios for the grid computing sites currently providing both computing and storage resources. In particular storage has been identified as the main challenge for HL-LHC due to the increasing volume of disk storage used, and also the costs from the site perspective to operate and maintain complex storage systems.\\
One of the goals of the working group is to propose and evaluate new scenarios for data access leveraging the potential benefits of a Data Lake model:
\begin{itemize}
\item Benefit from these new analysis data format and their reduction in size which are expected to be heavily accessed 
\item Reduction of number of site with Grid storage (e.g. stateless storage).
\item Provide efficient access to analysis datasets to diverse computing resources  (CPU, GPU, Machine Learning, etc.) to access the full analysis datasets.
\end{itemize}\\
One of the current approached is to investigate content delivery through caching layers infrastructures to minimize latency impact and increase file re-usability, at the site level or at regional level.
The engagement of the physics community will be crucial to converge on these new compact objects. %As the computing budget is foreseen to be flat, maximizing the efficiency of the storage resources will be mandatory to cope with the large amount of data necessary to maximise the physics potential of the HL-LHC program. \\
