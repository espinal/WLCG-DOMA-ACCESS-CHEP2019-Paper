\section{Conclusions}
The DOMA Access Working Group is providing input and recommendations about the possible future directions for addressing the data access challenges in the HL-LHC scenario from the user analysis perspective. In the upcoming year we will increase our understanding about the operations and performance of the different caching infrastructures that are running worldwide for ATLAS and CMS. We are also engaged with the physics analysis community and liaising with the HSF Analysis working group to understand the future needs from the physics community for the HL-LHC. The goal is to provide and prototype the infrastructures able to cater with the HL-LHC data processing demands.\\

The implications on the Data Lake storage infrastructure as a data source and its role in the experiment data workflows are the dominant factor. The DOMA ACCESS working group need to evolve and start addressing in detail the full picture from data storage, data distribution and data access to the combined impact of workloads and their requirements on the infrastructure.

 The experience and information gathered during the initial mandate of the working group will provide precious guidelines for this future work towards a new data storage infrastructure and new data processing models that should start being evaluated during LHC Run-III.
 
 The obtained results on content delivery with the different caching infrastructures confirm it as a promising mechanism to address the analysis challenge. This approach also promote an efficient use of the storage at the sites and hence help to optimize the overall storage cost while still meeting the HL-LHC data storage needs. 

