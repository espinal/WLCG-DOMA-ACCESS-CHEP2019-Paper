\section{Conclusions}
The DOMA Access Working Group is focusing the end of their mandate by December 2020. The goal is to wrap up on the investigations performed during the two years mandate (2019 and 2020) and provide input and recommendations about the possible future directions to address data access from the analysis data perspective. In this coming year we will learn more about the operations and performance of the different caching infrastructures running worldwide for ATLAS and CMS and we will also will have a clearer picture of experiments commitment for the envisaged compact data objects for analysis.\\
The implications on the datalake storage infrastructure as a data source and the data workflows are a dominant factor and hence the working group would need to evolve and start addressing in detail the storage consolidation in the form of datalakes and looking at the data access taking into account the full picture: detectors, data distribution and storage, experiment workflows and analysis workflows. The experience and information gathered during this initial mandate will provide precious guidelines for this future work towards a new data storage infrastructure and new data processing modelling to start being evaluated during Run-III.
