\documentclass{webofc}
\usepackage[varg]{txfonts}   % Web of Conferences font
\usepackage{url}
\begin{document}
\title{The Quest to solve the HL-LHC data access puzzle}
\author{\firstname{X.} \lastname{Espinal}\inst{1} \and
        \firstname{S.} \lastname{Jezequel}\inst{2} \and
        \firstname{M.} \lastname{Schulz}\inst{1} \and
        \firstname{A.} \lastname{Sciab\`a}\inst{1} \and
        \firstname{I.} \lastname{Vukotic}\inst{3} \and
        \firstname{F.} \lastname{Wuerthwein}\inst{4}
}

\institute{European Organisation for Nuclear Research (CERN), Geneva, Switzerland \and 
        LAPP, Université Grenoble Alpes, Université Savoie Mont Blanc, CNRS/IN2P3, Annecy; France. \and
        University of Chicago, Chicago, Illinois, US \and
        University of California, San Diego, La Jolla, CA, USA
}

\abstract{HL-LHC will confront the WLCG community with enormous data storage, management and access challenges. These are as much technical as economical. In the WLCG-DOMA Access working group, members of the experiments and site managers have explored different models for data access and storage strategies to reduce cost and complexity, taking into account the boundary conditions given by our community.
Several of these scenarios have been evaluated quantitatively, such as the Data Lake model and incremental improvements of the current computing model with respect to resource needs, costs and operational complexity.
To better understand these models in depth, analysis of traces of current data accesses and simulations of the impact of new concepts have been carried out. In parallel, evaluations of the required technologies took place. These were done in testbed and production environments at small and large scale.
We will give an overview of the activities and results of the working group, describe the models and summarise the results of the technology evaluation focusing on the impact of storage consolidation in the form of Data Lakes, where the use of read-ahead caches (XCache) has emerged as a successful approach to reduce the impact of latency and bandwidth limitation.
We will describe the experience and evaluation of these approaches in different environments and usage scenarios. In addition we will present the results of the analysis and modelling efforts based on data access traces of the experiments.}

\maketitle

\section{Introduction}
The WLCG strategy paper \cite{wlcg} set out the path towards computing for the HL-LHC era, building up from the input provided by the HSF \cite{hsf} Community White Paper \cite{cwp}.
The estimates for the data volumes and computing show a major step up from the current needs and a program of work was established from the WLCG point of view to address this future challenge. One of the charges is addressed by the DOMA access working group to evaluate future data access scenarios.\\
The working group collected information from experiments future plans about analysis data formats, is following-up the work pioneered by the cost model working group to understand file placement and file usage statistics and is investigating data caches infrastructures and promoting the deployment of caching models leveraging data access from a consolidated storage infrastructure labeled as datalake ~\ref{datalake-sketch-horizontal}.

\begin{figure}[h]
  \centering
  \includegraphics[height=5cm]{Datalake-sketch-horizontal.png}
  \caption{{\em} Conceptual sketch of the datalake idea}
  \label{datalake-sketch-horizontal}
\end{figure}





\section{Compact analysis objects}
CMS and ATLAS experiments are moving towards compact datasets for analysis with event sizes on the order of the kB/event. This implies that for CMS the full analysis datasets will around 1PB per year for this compact model named nanoAOD (c.f. with 50PB per year for miniAOD) taking as a basis the estimation of having 80(data)+160(MC) billion events/year and mean sizes for the different data types of 7.4MB(RAW), 2.0MB(AOD), 200kB(miniAOD) and 4kB(nanoAOD).
This new objects open the window to evaluate new ways of doing computing and different options for the sites providing computing and storage resources.\\
In particular storage has been identified as the main challenge for HL-LHC due to the increasing use of disk and the need for site operation and administration of complex storage systems. From the working group perspective one of the goals is to study the feasibility to exploit this new analysis objects and its reduction in size to promote less demanding storage at the sites (e.g. stateless storage) while fostering the efforts towards providing more efficient computing resources. The option to have a full analysis objects in a centralized storage while data access is performed through caches to minimize latency and increase file re-usability at site or regional level. At this point in time there is the need of the engagement from the physics community to converge using these objects but the goal is clearly to be able to maximise the physics potential of the machine and hence taking as much data as possible which means to maximise the use and efficiency of the global storage resources. Use of storage is a delicate concept, a full storage does not mean it is used in an efficient way.\\


\section{File usability and data access patterns}
One of the key parameters to assess our effective storage usage is to measure the access frequency after data placement. The two extremes regarding data thermodynamics are \emph{cold data} where files are WORN (Write Once Read Never) and \emph{hot data} where files are accessed quickly after data placement and with high concurrency. After studying data access patterns at several sites we observed that large fraction of our files are neither totally \emph{cold} nor \emph{hot}. The analysis files lose popularity with time and the access rate decreases significantly after days/weeks, in (Fig. \ref{access}) the file rates and file popularity on a Tier-1 and a Tier-2 are shown as a function of time as a representative example.

\begin{figure}[h]
  \centering
  \includegraphics[height=8cm]{dataaccess-chep2019.png}
  \caption{{\em (left)} File popularity on a Tier-1 and a Tier-2 as a function of time (300 days). The plots above indicate that data is not accessed very often, it is most likely to be re-read within days after placement then the access drops substantially, almost two orders of magnitude.}
  \label{access}
\end{figure}

This provides an indication whether this type of data could be better accessed through a cache, so it is available when is popular and gets super-seeded with newer files once they are less demanded. In this way the space on disk at the computing sites is optimised for data being actively used and this can potentially be completely delegated to an \emph{stateless} cache. In parallel less frequently used data might be re-fetched again from the Data Lake (disk or tape) where the experiments will handle the popularity with the required Quality of Service (QoS) to make use of the best cost/usage ratio for the storage.\\
We also observed a fundamental difference between analysis and production data. Analysis has higher re-use while production files have very few re-reads. As a result running combined workflows on a site has the effect to push analysis data out of the cache.

%This made us think that in the case we do not change much of the current infrastructure we would nevertheless benefit by changing the current model and favoring running predictable and time-defined workflows at the sites with less storage and favor less-predictable user analysis on sites with larger storage services.\\

It should be noted that these observations are based only on a period of six months but they provide hints towards a cache-oriented storage. Further studies should be done on longer period and also combined with staging and data deletion information. 


\section{Data caching: concept, infrastructure and initiatives}
Simulations of caching layers based on reference WLCG workloads showed the ability to hide latency 
even when data is read for the first time. The simulations have been conducted using using XCache technology (from the xrootd software framework \cite{xroot}).

Within the root framework \cite{root} it is also possible to cache data from the client side (reading ahead) while the file starts to be accessed; this is very effective for low latencies and enables the remote reading option for sites close to the Data Lake (storage-less). For higher latencies the impact starts to be noticeable and CPU inefficiencies grow with the increasing latency.\\
There are 160 computing facilities spread worldwide contributing to the WLCG infrastructure. They have different roles and scopes, supporting different experiments and local scientific communities, different analysis groups, etc. Among these many sites we have a big variety on network topologies and latencies, also they might have different scientific interests and plans concerning their future computing services and infrastructures, and in particular on storage services. One of the options we are exploring in the working group is to enable storage-less or state-less storage approach for sites that are interested mainly in processing facilities and liberate them to run a full fledged storage system. These sites will use the Data Lake as the main source of data and will access these data via caches or remote access form their data processing facilities.

%The new models for distributed computing together with the Data Lake concept will open the opportunity for sites accessing data from all sites from Data Lake. 
%An example could be a relatively small Tier-2 currently providing storage and computing 
%to WLCG experiments, needing to maintain a storage system which is of little use to them. If this site is close 
%enough to a the Data Lake they could consider about accessing data remotely, and if they are on a more distant 
%(in network latency units) place they could interface with data by deploying a stateless storage as a caching 
%layer for latency hiding and eventual file reusability. 
In Fig. \ref{datalake-sketch} is shown a tentative 
sketch envisioning a Data Lake composed by sites and federations holding the bulk of the data regions, and 
the different types of computing-oriented sites, commercial clouds and HPCs accessing the Data Lake.\\

\begin{figure}
  \centering
  \includegraphics[height=7.8cm]{datalake-sketch-square.png}
  \caption{{\em (center)} Data Lake sketch composed by sites and federations holding the bulk of the data regions, and the different types of computing-oriented sites, commercial clouds and HPCs accessing the Data Lake }
  \label{datalake-sketch}
\end{figure}
The working group has promoted the deployment of several caching models to operate in a region and on a site level. We are investigating three different approaches: a) High performance caching servers in US to feed a region Southern California (SoCal) and Chicago; and b) caching federation to feed data to regional sites (as in Italy); and c) a site caching mechanism as state-less Tier-2 storage (Munich and Birmingham). The results obtained confirm that caching is a promising mechanism to address the analysis challenge and help increasing an efficient usage of the storage and hence able to optimize the overall cost, still meeting the
HL-LHC data storage needs. The caching layer setup at SoCal demonstrates that three sites (Riverside, Caltech and San Diego) can benefit from a common caching layer of 1~PB (c.f. with the old model where the site had to deal with 5~PB of stateful storage installation), this cache can serve 90\% of the jobs/user request at 1/5th of the cost in hardware and alleviating the site to manage a complex storage service.\\
The initiative in LMU Munich demonstrated that an old disk pool node, with a simple hardware configuration (JBOD) and simple XCache deployment could serve up to 3k concurrent jobs of ATLAS workflows reading data from the neighbour site in Hamburg (DESY) and from a far site in China (IHEP in Beijing). The test concluded that the difference in CPU efficiency when reading from the neighboring site and from the far site is no longer a showstopper taking into account the distance and the latency (Fig. ~\ref{lmu-xcache}).\\

\begin{figure}[h]
  \centering
  \includegraphics[height=6.5cm]{lmu-xcache.png}
  \caption{{\em (center)} XCache running on modest hardware at LMU. Successfully served 3.2k analysis and derivation jobs from ATLAS with and average I/O of 1~MB/s and 3~MB/s respectively. Effective latency hiding is achieved for high latency data consumption.}
  \label{lmu-xcache}
\end{figure}






\section{Conclusions}
The DOMA access working group is focusing the end of their mandate by December 2020. The goal is to wrap up on the investigations performed during the two years mandate (2019 and 2020) and provide input and recommendations about the possible future directions to address data access from the analysis data perspective. In this coming year we will get more input from the different caching infrastructures running worldwide for ATLAS and CMS and also will have a clearer picture of experiments commitment for the envisaged compact data objects for analysis. We do see that the implications of the data source are dominant and hence we feel the working group would need to evolve and start addressing storage consolidation in the form of datalakes and looking data access taking into account the full picture: detectors, data distribution and storage, experiment workflows and analysis workflows. We do think the experience and information gathered during this initial mandate will provide precious guidelines for this future work towards a new data storage infrastructure and new data processing modelling to start being evaluated during Run-III.

\urlstyle{same}
\begin{thebibliography}{}
\bibitem{wlcgstrategy}
I.~Bird, S.~Campana \textit{https://cds.cern.ch/record/2621698}
\bibitem{hsf}
HEP Software Foundation, \url{https://hepsoftwarefoundation.org/}
\bibitem{cwp}
HEP Software Foundation, \textit{A Roadmap for HEP Software and Computing R\&D for the 2020s}, arXiv:1712.06982 (2018)
\bibitem{nano}
A. Rizzi, G. Petrucciani and M. Peruzzi for the CMS Collaboration \textit{Further reduction in CMS event data for analysis: the NANOAOD format} EPJ Web of Conferences 214, 06021 (2019)
\bibitem{xroot}
A.~Hanushevsky {\em et al}, \url{https://xrootd.slac.stanford.edu/}
\bibitem{root}
Rene Brun and Fons Rademakers, ROOT - An Object Oriented Data Analysis Framework,
Proceedings AIHENP'96 Workshop, Lausanne, Sep. 1996, Nucl. Inst. \& Meth. in Phys. Res. A 389 (1997) 81-86. See also [root.cern.ch/](http://root.cern.ch/).
D.~Lange {\em et al}, \textit{CMS Computing Resources: Meeting the demands of the high-luminosity LHC physics program}, these proceedings
\bibitem{costmodel}
R.~Vernet,J.\ Phys.: Conf.\ Ser.\ \textbf{664}, 052040 (2015)
\bibitem{wlcg}
Worldwide LHC Computing Grid, \url{http://wlcg.web.cern.ch}
\end{thebibliography}



\end{document}
